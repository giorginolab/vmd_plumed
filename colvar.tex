%% This template can be used to write a paper for
%% Computer Physics Communications using LaTeX.
%% For authors who want to write a computer program description,
%% an example Program Summary is included that only has to be
%% completed and which will give the correct layout in the
%% preprint and the journal.
%% The `elsarticle' style is used and more information on this style
%% can be found at 
%% http://www.elsevier.com/wps/find/authorsview.authors/elsarticle.
%%
%%
\documentclass[preprint,12pt]{elsarticle}

%% Use the option review to obtain double line spacing
%% \documentclass[preprint,review,12pt]{elsarticle}

%% Use the options 1p,twocolumn; 3p; 3p,twocolumn; 5p; or 5p,twocolumn
%% for a journal layout:
%% \documentclass[final,1p,times]{elsarticle}
%% \documentclass[final,1p,times,twocolumn]{elsarticle}
%% \documentclass[final,3p,times]{elsarticle}
%% \documentclass[final,3p,times,twocolumn]{elsarticle}
%% \documentclass[final,5p,times]{elsarticle}
%% \documentclass[final,5p,times,twocolumn]{elsarticle}

%% if you use PostScript figures in your article
%% use the graphics package for simple commands
%% \usepackage{graphics}
%% or use the graphicx package for more complicated commands

\usepackage[utf8]{inputenc}
\usepackage{graphicx}
\usepackage{subfigure}

%% or use the epsfig package if you prefer to use the old commands
%% \usepackage{epsfig}

%% The amssymb package provides various useful mathematical symbols
%% \usepackage{amssymb}
%% The amsthm package provides extended theorem environments
%% \usepackage{amsthm}

%% The lineno packages adds line numbers. Start line numbering with
%% \begin{linenumbers}, end it with \end{linenumbers}. Or switch it on
%% for the whole article with \linenumbers after \end{frontmatter}.
%% \usepackage{lineno}

%% natbib.sty is loaded by default. However, natbib options can be
%% provided with \biboptions{...} command. Following options are
%% valid:

%%   round  -  round parentheses are used (default)
%%   square -  square brackets are used   [option]
%%   curly  -  curly braces are used      {option}
%%   angle  -  angle brackets are used    <option>
%%   semicolon  -  multiple citations separated by semi-colon
%%   colon  - same as semicolon, an earlier confusion
%%   comma  -  separated by comma
%%   numbers-  selects numerical citations
%%   super  -  numerical citations as superscripts
%%   sort   -  sorts multiple citations according to order in ref. list
%%   sort&compress   -  like sort, but also compresses numerical citations
%%   compress - compresses without sorting
%%
%% \biboptions{comma,round}

% \biboptions{}

%% This list environment is used for the references in the
%% Program Summary
%%
\newcounter{bla}
\newenvironment{refnummer}{%
\list{[\arabic{bla}]}%
{\usecounter{bla}%
 \setlength{\itemindent}{0pt}%
 \setlength{\topsep}{0pt}%
 \setlength{\itemsep}{0pt}%
 \setlength{\labelsep}{2pt}%
 \setlength{\listparindent}{0pt}%
 \settowidth{\labelwidth}{[9]}%
 \setlength{\leftmargin}{\labelwidth}%
 \addtolength{\leftmargin}{\labelsep}%
 \setlength{\rightmargin}{0pt}}}
 {\endlist}

\journal{Computer Physics Communications}


%\newcommand{\mytitle}{PLUMED-GUI: a visual environment to develop
%  molecular dynamics analysis and biasing scripts}

\newcommand{\mytitle}{PLUMED-GUI: an environment for the interactive development 
  of molecular dynamics analysis and biasing scripts}
\newcommand{\mykeywords}{Graphical User Interface \sep VMD \sep PLUMED \sep Molecular Dynamics \sep Collective Variables \sep Metadynamics}

\begin{document}

\begin{frontmatter}

%% Title, authors and addresses

%% use the tnoteref command within \title for footnotes;
%% use the tnotetext command for the associated footnote;
%% use the fnref command within \author or \address for footnotes;
%% use the fntext command for the associated footnote;
%% use the corref command within \author for corresponding author footnotes;
%% use the cortext command for the associated footnote;
%% use the ead command for the email address,
%% and the form \ead[url] for the home page:
%%
\title{\mytitle}
%% \tnotetext[label1]{}
\author{Toni Giorgino\corref{cor1}\fnref{label2}}
\ead{toni.giorgino@isib.cnr.it}
%% \ead[url]{home page}
%\fntext[label2]{Footnote}
\cortext[cor1]{To whom correspondence should be addressed}
\address{Institute of Biomedical Engineering (ISIB),\\ 
National Research Council of Italy (CNR),\\
Padua, Italy}
%\fntext[label3]{Label3}

%%\title{A \LaTeX{} template for CPC Computer Physics Descriptions}
%% use optional labels to link authors explicitly to addresses:
%% \author[label1,label2]{<author name>}
%% \address[label1]{<address>}
%% \address[label2]{<address>}

%% \author[a]{First Author\corref{author}}
%% \author[a,b]{Second Author}
%% \author[b]{Third Author}

%% \cortext[author] {Corresponding author.\\\textit{E-mail address:} firstAuthor@somewhere.edu}
%% \address[a]{First Address}
%% \address[b]{Second Address}

\begin{abstract}
  This paper presents PLUMED-GUI, an interactive environment to
  iteratively develop and test complex Plumed scripts within the
  Visual Molecular Dynamics (VMD) environment. Computational
  biophysicists can take advantage of the best of both systems,
  leveraging a richer syntax and interactive operations, such as VMD's
  chemically-based atom selection language.  Support for inserting
  well-known reaction coordinates is provided through pre-defined
  templates and syntax mnemonics. Complex CVs, such as these involving
  snapshots for RMSD or native contacts calculations, can be built
  through guided dialogs grouping the available options.  Scripts can
  be either exported for use in simulation programs, or readily
  evaluated on the currently-displayed molecular trajectories, without
  leaving VMD, thus supporting an incremental try-see-modify 
  development cycle for molecular metrics.
\end{abstract}

% whose features include the ability to load and visualize
%  and manipulate multiple molecules and long trajectories.  

% its
%   scripting language, one can compute a wide range of collective
%   variables on existing molecular dynamics trajectories, or bias
%   simulations according to protocols such as metadynamics and
%   steering

%  make it an ideal environment to visualize and select parts of   complex molecular systems;

\begin{keyword}
\mykeywords
\end{keyword}

\end{frontmatter}

%%
%% Start line numbering here if you want
%%
% \linenumbers

% Computer program descriptions should contain the following
% PROGRAM SUMMARY.

{\bf Program summary}
  %Delete as appropriate.

\begin{small}
\noindent
{\em Manuscript Title:}                                       
 \mytitle \\
{\em Authors:}                                                
 Toni Giorgino \\
{\em Program Title:}                                          
 VMD-PLUMED (Collective variable analysis plugin) \\
{\em Journal Reference:}                                      \\
  %Leave blank, supplied by Elsevier.
{\em Catalogue identifier:}                                   \\
  %Leave blank, supplied by Elsevier.
{\em Licensing provisions:}                                   
 3-clause BSD Open Source. \\
{\em Programming language:}                                   
 TCL/TK. \\
{\em Computer:}                                               
 Should run  PLUMED [1] and VMD [2]. \\
{\em Operating system:}                                       
 Linux/Unix, OSX, Windows. \\
{\em RAM:}                                               
 Sufficient to run PLUMED [1] and VMD [2]. \\
{\em Number of processors used:}                              
 1 \\
% {\em Supplementary material:}                                 \\
  % Fill in if necessary, otherwise leave out.
{\em Keywords:} \mykeywords \\
  % Please give some freely chosen keywords that we can use in a
  % cumulative keyword index.
{\em Classification:}                                         
  3 Biology and Molecular Biology, 23 Statistical Physics and Thermodynamics. \\
% {\em External routines/libraries:}                            \\
  % Fill in if necessary, otherwise leave out.
{\em Subprograms used:}                                       
  PLUMED (version 1.3 or higher). \\
% {\em Catalogue identifier of previous version:}*              \\
  %Only required for a New Version summary, otherwise leave out.
% {\em Journal reference of previous version:}*                  \\
  %Only required for a New Version summary, otherwise leave out.
% {\em Does the new version supersede the previous version?:}*   \\
  %Only required for a New Version summary, otherwise leave out.
  {\em Nature of problem:} Compute and visualize values of collective
  variables on molecular dynamics trajectories from within VMD, and
  interactively develop biasing scripts for the estimation of
  free-energy surfaces in Plumed.
  \\
  {\em Solution method:} A graphical user interface is integrated in
  VMD and allows to interactively develop and run analysis scripts.
  Menus and dialogs provide mnemonics and documentation on the syntax
  to define complex CVs.
  \\
%{\em Reasons for the new version:}*\\
  %Only required for a New Version summary, otherwise leave out.
%{\em Summary of revisions:}*\\
  %Only required for a New Version summary, otherwise leave out.
  {\em Restrictions:}
  Tested on systems up to 100,000 atoms. \\
  {\em Unusual features:} VMD-PLUMED is not a standalone program but a
  plugin that provides access to PLUMED's analysis features from within VMD. \\
  {\em Additional comments:} Distributed with VMD since version 1.9.0.
  Manual  update may be required  to access the latest features.   \\
  {\em Running time:} Depends on the size of the system and the length
  of the trajectory; usually negligible with respect to simulation time.  \\
\begin{thebibliography}{0}
\bibitem{1}See paper in this issue.
\bibitem{2}Humphrey, W., A. Dalke, and K. Schulten. ``VMD: Visual Molecular Dynamics.'' J Mol Graph 14, no. 1 (February 1996): 33-38. 
\end{thebibliography}

%* Items marked with an asterisk are only required for new versions
%of programs previously published in the CPC Program Library.\\
\end{small}



\section{Introduction}


 Plumed's scripting
  language can express a variety of collective variables and
  force-biasing protocols, which can be applied either to the analysis
  of existing trajectories, or to bias new simulations.

 PLUMED-GUI
  streamlines the development and test of analysis scripts, 


Developing reaction coordinates is a central task

Several analysis frameworks were developed

Plumed is used to bias systems. Has an expressive language 

As a byproduct, the same definitions can be used to evaluate
existing, precomputed trajectories, thus making it  a good analysis
framework


VMD's builtins

None of these is interactive







- large scale simulations becoming commonplace

- reactions can be described as long as one has good reaction coordinates

- RC can be used post-hoc for analysis

- need methods to test collective variable definitions

- when one has a trajectory, he wants to be quickly get the value of given CVs over it

- CV can be quite sophisticated / see plumed 

-- however, they have to be defined with chemical awareness

-- this implies taking into account topologies

-- for example expressing lists of atoms as quasi-natural-language expressions

-- DRIVER


\section{Plugin usage}

The PLUMED-GUI interface is opened selecting the
\emph{Analysis/Collective variable analysis (PLUMED)} entry found in
VMD's \emph{Extensions} menu.  The interface (Figure~\ref{fig:main})
behaves  as a text editor; \emph{File} and \emph{Edit} menus
provide  customary editing commands, including  open and save,
copy/paste and undo/redo operations.  The Plumed script is entered in the main
text area.   When started, the text area displays a
brief syntax reminder, which can be dismissed. 

Pressing the \emph{Plot} button at the bottom of the window evaluates
the displayed script to the currently selected trajectory (known
within VMD as the all-important \emph{top} molecule). Assuming that
Plumed executable is properly installed, the GUI will execute Plumed's
\emph{driver} function, which will evaluate the values of the CVs
defined in the script at each of the top trajectory frames.

%It is worth noting that the editor does not enforce syntax checks. 

It is worthwhile noting that the GUI does not restrict the syntax of
the input; the script is passed as-is to the underlying Plumed engine,
with the sole exception of expanding symbolic atom selections in
square brackets, as will be shown in
Section~\ref{sec:symb-atom-select}.  Script coding and debug is
entirely under the control of the user, and any valid or invalid
expression can be entered.  (Consequently, the GUI does not need to be
updated to support user-customized Plumed versions and future syntax.)
In general, the script should follow the syntax of the Plumed version
currently in use.

Should the evaluation of the script generates an error, it will be
displayed in the VMD textual console.  In most instances Plumed
identifies the specific problem and corresponding script line; when this
happens, the error line will be marked as such
in the text area.

% Once the evaluation is complete, a plot is displayed with the time
% series of the collective variables, each shown with a different color.


\subsection{Analysis and visualization}

Once the evaluation is successful, the time series of the collective
variables are displayed graphically in a plot.  The purpose of the
plot is to quickly inspect the values yielded by the current CV
definitions, and provide a way to iteratively refine them. The plot
layout is basic, showing time on the abscissa, and the CV values in
different line colors; data points can be optionally read out hovering the
mouse pointer.  More complex visualizations can be obtained exporting
data to external plotting programs; data can be exported 
either as a matrix (time running as rows, and CVs as
columns), or as consecutive time-value vectors separated by empty
lines.




\begin{figure}
  \centering
  \includegraphics[width=0.8\textwidth]{images/main}
  \caption{PLUMED-GUI's main window.  The analysis script is entered
    in the text area, like a text editor; shortcuts accessible from
    the \emph{Templates} menu insert frequently-used definitions. The
    \emph{Plot} button evaluates the collective variables defined in
    the script on the molecular trajectory currently selected in VMD
    (``top molecule''); if successful, a graph of the values of the
    CVs at each frame is shown. The gray box, only shown at startup,
    is a brief reminder about the use of the interface.  }
  \label{fig:main}
\end{figure}




\subsection{Consistency of units}

It may be worth noting that the units of computed CVs  depends on
Plumed's conventions.  Since version 2.0, Plumed defaults to the nm,
kJ/mol, ps combination. Given that VMD users may be accustomed to the
\AA, kcal/mol, fs unit set, a reminder is shown about the fact that
the \texttt{UNITS} keyword can be used to  switch to customary
units.



\section{Assistance to Editing}


\subsection{Symbolic atom selections}\label{sec:symb-atom-select}

VMD contains a syntax for expressing atom selections; strings such as
\texttt{same residue as (protein or water within 4 of name CA)} are valid
expressions that are interpreted at run time, and return a list of
atoms.  An extensive set of keywords exists to query numerical
(coordinates, beta values, residue IDs), as well as chemical
(e.g. polar, atom names) and many other properties, as documented
elsewhere~\cite{Humphrey_Dalke_Schulten_1996}.

PLUMED-GUI allows  embedding of VMD's atom selections in analysis scripts
through  square brackets. As shown in Figure~\ref{fig:main},
textual expressions in square brackets are evaluated with respect to
the top molecule and frame, and transparently replaced with the
resulting list of atoms. In this way, Plumed users can 
avoid the use of numeric atom IDs altogether in favor of human-readable
alternatives such as \texttt{[protein and name CA]}.

This is especially advantageous when switching between multiple
systems, with similar, not exactly equal topologies; this is the case,
for example, when several all-atom systems are solvated and
parametrized with the same protein but a series of inhibitors.  Serial
numbers  depend on the specific system and the details of how it was prepared, but
symbolic selections such as \texttt{[not protein and not water]}
(matching non-peptide ligands) will be valid
regardless of the specific ligand being analyzed.

Symbolic atom expressions are interpreted the moment the analysis is
requested by pressing the \emph{Plot} button. They can also be
permanently expanded for use independently of PLUMED-GUI, via the
\emph{Export} function shown in section~\ref{sec:export-use-simul}.



\subsection{Templates}

The \emph{Templates} menu provides access to a number of frequently-used
definitions; selecting one of the menu entries inserts the
corresponding keyword (Figure~\ref{fig:templates}). Templates, in
other words, offer human-readable shortcuts to enter the frequently used
strings that define atom groups and CVs.

After selection, inserted templates can be edited freely in the text
area.  Some portions to be manually filled in; for example, in the
case of the ``Coordination'' template, one has to specify one or two
groups between which the coordination number is to be computed, and
the parameters of the switching function.

The list of templates provided in the menu is not meant to be
exhaustive, but to provide access to the most frequently-used CVs;
only the default options are usually inserted. Generic actions and
modifiers can be typed manually, while optional keywords can be looked
up through an on-line contextual help.

\begin{figure}
  \centering
  \includegraphics[scale=.5]{images/templates}
  \caption{The \emph{Templates} menu contains shortcuts that
    insert frequently-used definitions and collective variables.}
  \label{fig:templates}
\end{figure}

\subsection{On-line help}

Plumed's actions have a wealth of options to further specify the
behavior of CVs. For instance, the \texttt{COORDINATION} action
foresees modifiers to define the shape and functional form of the
switching function; to ignore periodic boundary conditions; to compute
derivatives numerically; and several others. The richness of the
syntax may make it unwieldy to recall the syntax of lesser-used
keywords.

% Although templates include placeholders for most
% common options, it would be impractical to include all of the optional
% parameters in each template line.

PLUMED-GUI provides a comprehensive context-dependent help facility
through a pop-up menu, which is be invoked pressing the right mouse
button on any action keyword (Figure~\ref{fig:help_popup}). The pop-up
menu shows the list of optional and mandatory modifiers accepted by
that action.  The topmost menu item, \emph{Lookup in documentation},
opens up a web browser  displaying the full manual page of that action
(Figure~\ref{fig:help_page}).

As for the rest of Plumed 2.0 documentation, PLUMED-GUI's contextual
help is generated automatically from the source code.  This implies
that, as long as new features are implemented and documented following
the coding conventions, the new functions become properly integrated
in the GUI.


% This implies
% that, as long as the coding conventions are followed in the
% implementation and documentation of new features, the new functions
% will be properly integrated in the GUI.




\begin{figure}
  \centering
  \subfigure[Contextual help]{
    \includegraphics[width=0.45\textwidth]{images/help_popup}
    \label{fig:help_popup}
  }
  \subfigure[Sample manual page]{
    \includegraphics[width=0.40\textwidth]{images/help_page}
    \label{fig:help_page}
  }    
  \caption{(a) A contextual popup menu lists mandatory and
    optional keywords supported by the action under the pointer (in
    this case, \texttt{COORDINATION}, which computes the coordination
    number of one or two groups of atoms). (b) The \emph{Lookup} function
    displays an action's manual in a web browser. }
  \label{fig:help}
\end{figure}





\section{Structure-based functions}

%The \textit{Structure} menu supports the construction of complex CVs.
%Some CVs require too much information to be defined by one script command.

Functions in the \emph{Structure} menu provide assistance in the
declaration of CVs that depend upon the topology and coordinates of
the currently loaded system.  Each of the menu entries opens up a
dialog with the tunable options for the CV being defined. Structure-based CVs
generally involve long lists of statements and/or auxiliary files;
automated procedures therefore relieve users from the error-prone
process of building these lists by hand.



\begin{figure}\label{fig:structure}
  \centering
  \subfigure[Build reference structure]{
    \includegraphics[scale=.45]{images/reference} \label{fig:reference}}
  \subfigure[Native contacts]{
    \includegraphics[scale=.45]{images/native} \label{fig:native}}
  \subfigure[Ramachandran angles]{
    \includegraphics[scale=.45]{images/rama} \label{fig:rama}}
  \caption{Dialogs accessible from the \emph{Structure} menu support
    the creation of CVs based on the active topology. (a) \emph{Build
      reference structure} converts the currently displayed frame into
    a reference file for RMSD calculations.  Atom sets to be used for
    alignment and displacement are specified as VMD atom selections;
    numbering can also be mapped between molecules if the reference
    frame and the trajectory on which the CV will be computed belong
    to systems with different topologies. (b) Analogously,
    \emph{Native contacts} enumerates  the atom pairs (closer
    than the chosen threshold distance) in the currently-displayed
    (``native'') frame. The CV will measure how many of
    those atom pairs will present in each trajectory frame. 
    Non-informative contacts between neighboring residues can be
    filtered out putting a lower bound to the $| \Delta \mbox{resid}
    |$ parameter. (c) The \emph{Ramachandran angles} dialog inserts
    CVs corresponding to $\phi, \psi$ and/or $\omega$ backbone
    dihedrals contained in the selection.}
\end{figure}

\subsection{Generating reference structures for alignments}

% A typical example of a CV which requires structural data for its
% definition, and thus more information than contained in the script, is
% the root mean square deviation (RMSD) of a set of atoms with respect
% to a reference frame. 

The root mean square deviation (RMSD) metric is frequently used to
detect structural similarities and conformational transitions.  RMSD
values are computed averaging the squared displacement of a chosen set
of atoms (displacement set) with respect to a \emph{reference}
structure, computed after applying the roto-translation that optimally
aligns another, possibly coincident, set of atoms (alignment set).  In
addition to the usual definition, Plumed implements a generalization
of the metric, namely the $S$ and $Z$ path variables, which express
the ``progression'' and ``distance'' of the current state of the
system along a path defined by an arbitrary number of exemplary
reference structures used as
landmarks~\cite{Branduardi_Gervasio_Parrinello_2007}.


% A simplified procedure to generate reference frames from loaded
% structures is therefore especially useful.

The set of atoms to be used for alignment and computing the
displacement, and the reference coordinates are defined through
``reference files''; these are PDB-like text files, wherein each line
represents one atom involved in the calculation, and columns indicate
serial numbers, coordinates, and inclusion in one or the other set.
The \emph{Build reference structure} dialog
(Figure~\ref{fig:structure}) provides a convenient way to generate
such reference files.  This function ``freezes'' the coordinates of the
\emph{currently selected} frame into a reference file. The subset of
the atoms that will be involved, respectively, in the computation of
the optimal alignment, and the measure of the RMSD, are specified as
atom selections.

% The reference structure is the taken as the current frame of the top molecule. 

By default, the reference file generated is suitable for computing
RMSD on systems having the same topology as that from which the
reference was extracted.  However, it is sometimes necessary to
perform alignments between different topologies: the native structure
may for instance be a PDB file, while the system under analysis is the
all-atom structure used in simulation. Alignments between molecules
with different topologies are accommodated setting the \emph{target
  molecule ID}. This feature adjusts the atom numbering of the top
molecule to be compatible with the specified \emph{target} molecule;
in other words, trajectory frames of the target molecule will be
aligned with the structure of the top molecule, even though the
topologies of the two are different. The renumbering feature requires
that the atom selections match the same number of atoms in the two
systems.





\subsection{Number of native contacts}

The number of native contacts is a metric  frequently used to
determine structural similarity, especially as an indicator of folding
or binding.  The metric puts the accent on the presence of those
contacts that characterize the desired (native) structures. First, the
pairs of atoms in contact in a native structure are
enumerated, and a list of such pairs is built. Then, this list is evaluated
for each of the trajectory frames under analysis: the 
CV counts how many of the contact pairs that were present in the
reference frame are also present in the frame being analyzed.  

The \emph{native contacts} dialog (Figure~\ref{fig:native}) can be
used to generate such lists flexibly and with ease. Like when building
reference structures, the current frame of the top molecule is used as
the native state.  It is possible to specify either one or two atom
selections; in the first case, the contacting pairs involving atoms in
the selection are enumerated; otherwise, if two selections are given,
intermolecular contacts -- bridging the two selections -- will be
counted.  The ``distance cutoff'' box adjusts the distance (in \AA) at
which an atom pair is assumed to be in contact.

A marked rise in the number of native contacts is often used as a
proxy for the detection of folding events. However, residues adjacent
in the primary sequence will almost always be in contact, thus
contributing little or no information to the folding signal. These
``spurious'' contacts can be filtered out setting a minimum bound to
the $| \Delta \mbox{resid} |$ to a positive integer $\delta$. If set,
contacts between atoms closer than $\delta$ residues apart in the
primary sequence will be disregarded.

Analogously to the \emph{Build reference structure} function, the user
can match a trajectory with a native frame with a different topology
specifying the appropriate target molecule ID.

The number of native contacts is implemented in Plumed through the
\texttt{COORDINATION PAIRS} action and the enumeration of the
contacting pairs in the native frame.  It is worthwhile noting that,
like all other CVs provided by Plumed, this metric is a continuous
approximation of the original definition. In particular, the number of
contacts is made smooth in all the system's coordinates through
exponential averaging.




\subsection{Ramachandran angles}

The \emph{Ramachandran angles} dialog allows the insertion of the
standard backbone torsion angles defined between neighboring residues.
The user is asked to specify an atom selection. When the \emph{Insert}
button is pressed, CVs will be inserted to compute those $\phi$,
$\psi$ and/or $\omega$ angles that are contained in the
selection. Each angle will be a separate CV computed through the
appropriate \texttt{TORSION} keyword, defined according to the IUPAC
rules for biochemical nomenclature~\cite{IUPAC}. The generated CVs
include comments identifying the involved residues in a human-readable
way.








\section{Export for use in simulation}\label{sec:export-use-simul}

Plumed has extensive facilities to apply force biases to molecular
dynamics simulations with the objective to enhance the sampling of the
phase-space in a way that allows the reconstruction of free-energy
surfaces. Example of biasing protocols include harmonically
constraining CVs at a given combination of values (used e.g. in the
umbrella sampling protocol, \cite{Torrie_Valleau_1977}), pulling them
towards increasing or decreasing values (steered MD,
\cite{Isralewitz_Gao_Schulten_2001,Giorgino_2011}), metadynamics
\cite{Laio_Parrinello_2002}, and so on. Biased MD simulations are
carried out in codes patched to embed the Plumed engine. Force biases
are specified in the plumed script, which defines the CVs to be biased
and the biasing protocol.  Atoms have to be specified through their
serial numbers, which makes the iteration of complex scripts through
different systems an error-prone exercise.

The \emph{Export} function, accessible from the File menu, removes all
the symbolic atom selections in the current script, replacing them
with the corresponding numerical lists. The exported script is thus
independent of VMD-specific constructs, and can then be employed for
simulations.  The exported file contains comments to remind how the
numeric atom lists were computed. However, for the sake of
reproducibility, it is generally advisable to keep the original
script with unsubstituted, symbolic atom selections.



\section{Installation and compatibility}

The GUI supports the same wide range of platforms as VMD, encompassing
all major variants of Linux/Unix, OSX, and Windows.  Trajectory
analysis is performed invoking the platform-specific \emph{driver}
executable. Plumed distributions provide instructions on how to build
the executable on Unix-like systems; a precompiled version for Windows
is provided for convenience.

The current version of the plugin, PLUMED-GUI 2.0, supports both
Plumed 1.3 and Plumed 2.0, with minor functional differences.  In
particular, the language syntax and \emph{driver} invocation method
differ between the two.  The GUI detects which version is installed
upon start, and adapts templates and syntax accordingly.  If both
plumed versions are available, the user can switch manually between
the two. For clarity, this paper only focuses on the features
available in combination with Plumed 2.0.

Recent VMD distributions contain a preinstalled version of PLUMED-GUI.
This paper describes version 2.0 of the GUI, which supports Plumed 2.0
and earlier.  Users may manually update their GUI to the latest
version.  To update the GUI, it is sufficient to identify VMD's
installation path, and replace the files contained in the subdirectory
\texttt{plugins/noarch/tcl/plumed} with those contained in the latest
distribution of PLUMED-GUI.  As customary, the \emph{About} menu item
displays the currently installed version of the GUI.





% \subsection{Compatibility} 

% \subsection{Version and updates}



% To analyze
% trajectories, the plugin executes Plumed's ``driver'' binary for the
% platform; on Unix-like systems, such binary is obtained following the
% instructions included in Plumed's distribution; 







\section{Conclusions}

Searching for an appropriate combination of reaction coordinates is a
central task for the analysis of biomolecular systems.  This paper
described a graphical user interface developed to simplify the
iterative development and test of collective variables to be used with
the Plumed engine. The interface is hosted in a state-of-the-art
molecular visualization program,
VMD~\cite{Humphrey_Dalke_Schulten_1996}, leveraging its ability to
load and manipulate multiple systems long trajectories
concurrently, its extensive visualization options, and a powerful
chemically-aware atom selection language. The mathematical computation
of the CVs is demanded to the Plumed engine, which
provides a uniform, straightforward syntax to express a wide range of
collective variables used in the practice of biased molecular simulations.

PLUMED-GUI bridges the usability of a graphical interface and the
richness of Plumed's CV description language.  The integration of the
two environments has a few limitations; right now, only orthorhombic
simulation boxes with constant edges are supported, thus ruling out
the analysis of constant-pressure simulations (this limitation may be
removed as Plumed's support to trajectory formats is
expanded). Another drawback concerns the use of atom selections: these
are evaluated only once in the GUI, and thus it is not possible to
have time-varying atom sets, nor they are supported by the Plumed
engine.  If analysis protocols involving time-varying atom sets are
desired, it will be necessary to resort to conventional VMD-TCL
scripting. It is worthwhile noting, however, that Plumed's continuous
switching functions (such as \texttt{DISTANCES LESS\_THAN})
provide sensible approximations to tasks such as counting the number
of atoms satisfying a given property.


METAGUI \cite{Biarnes_Pietrucci_Marinelli_Laio_2012}.





\section{Acknowledgements}

Part of the work described in this paper was conducted while at the
Computational Biophysics Laboratory at the Universitat Pompeu Fabra
(Barcelona). Support from the ``Beatriu de Pin\'os'' scheme of the
Generalitat de Catalunya is gratefully acknowledged.




%% References with bibTeX database:

\bibliographystyle{elsarticle-num}
\bibliography{colvar}



\end{document}

;; %%% Local Variables:
;; %%% TeX-command-default: "Make"
;; %%% mode: latex
;; %%% TeX-master: "masterfile"
;; %%% End: 


